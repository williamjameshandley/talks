\section{CONCLUSION}
\label{sec:conclusion}

In this work, we presented a Bayesian framework for the automated processing of Type Ia supernova light curves. This approach builds upon and extends the comprehensive data curation methods employed in the Hawaii Supernova Flows dataset, offering a fully automated alternative to manual inspection that becomes essential for future large scale surveys. We integrate an anomaly detection model directly into the likelihood function, treating the quality of each data point as a latent variable. This allows for simultaneous inference of the physical supernova model and probabilistic identification of data contamination, thereby reducing the reliance on manual preprocessing.

Our application to the high quality data from the Hawaii Supernova Flows survey demonstrated that this framework addresses several challenges in SNe Ia analysis. Firstly, it provides a mechanism for mitigating the influence of individual outliers on parameter estimates. Secondly, it automates filter selection by identifying systematically corrupted bandpasses in a reproducible manner. Thirdly, the framework facilitates improved data preservation by flagging specific anomalous epochs, which retains valid data points within a filter that might otherwise be discarded by conventional cuts. The statistical agreement (98.2 per cent of all SALT3 parameter estimates agreeing within their combined 1$\sigma$ uncertainty) demonstrates the reliability of our framework. 

Beyond identification, our contamination analysis quantified the systematic biases introduced by anomalous data. We found that anomalous points tend to be systematically brighter than model predictions ($C_{\mathrm{bright}}$ = 0.718 $\pm$ 0.862) and preferentially affect blue bandpasses ($C_{\mathrm{colour}}$ = -0.563 $\pm$ 0.989). If left uncorrected, these biases would lead to underestimated extinction and systematically lower distance moduli through the Tripp relation, potentially affecting dark energy constraints and Hubble constant measurements. These wavelength dependent contamination effects provide insight into the systematic differences observed in the colour parameter between our framework and traditional methods. The heterogeneous nature of these contamination effects across the sample validates our approach of object by object anomaly detection rather than blanket corrections.

This automated and statistical approach is well suited for the challenges of upcoming cosmological surveys. The data volume from the Vera C. Rubin Observatory's Legacy Survey of Space and Time will make manual inspection impractical, necessitating scalable and objective methods. Our framework offers one such pathway for processing large numbers of light curves, aiming to preserve the integrity and statistical power of the resulting cosmological samples, which will be essential for improving the precision of measurements that may help resolve the Hubble tension. While demonstrated for SNe Ia analysis with the SALT3 model, the underlying Bayesian methodology is general. It could be readily adapted for use with other SNe Ia light curve models, such as BayeSN \citep{Thorp2022} or SNooPy \citep{Burns2011}, or to other astrophysical domains where data may be affected by non Gaussian noise or contamination.

Ultimately, this work represents a crucial step towards building fully autonomous, end-to-end analysis pipelines for precision cosmology. Although demonstrated here with the SALT3 model, the framework's model-agnostic design makes it a versatile tool that can enhance any likelihood-based SNe Ia analysis, offering a pathway to more precise and reliable results across different modelling approaches. Subsequent work will assess the cosmological implications of these results by applying this refined dataset to a full cosmological parameter inference, quantifying how the removal of these systematic biases impacts key measurements such as the dark energy equation of state and the Hubble constant. By embedding data quality assessment within the core statistical model, we can move towards a more robust, efficient, and powerful era of cosmological discovery.

