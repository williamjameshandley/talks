\documentclass[%
    handout
]{beamer}
\usepackage{graphicx} % For including single page pdfs
\usepackage{bm}       % bold math
\usepackage{pgffor}   % for loop
\usepackage{tikz}
\usepackage{multimedia}
\usepackage{layouts}
\usepackage{hyperref}
\usepackage{cambridge_lecture}

\usepackage{calculator}
\usepackage{pgffor}   % for loop
\usepackage{tikz}

\newcommand{\cols}[3][0.5]{%
    \SUBTRACT{1.}{#1}{\wdthb}
    \begin{columns}
        \begin{column}{#1\textwidth}
            #2
        \end{column}
        \begin{column}{\wdthb\textwidth}
            #3
        \end{column}
    \end{columns}
}

\newcommand{\figname}{}
\newenvironment{figright}[2][0.5]{%
    \renewcommand{\figname}{#2}
    \SUBTRACT{1.}{#1}{\wdthb}
    \begin{columns}
        \begin{column}{#1\textwidth}
        }{%
        \end{column}
        \begin{column}{\wdthb\textwidth}
            \includegraphics[width=\textwidth]{\figname}
        \end{column}
    \end{columns}
}

\newcommand{\dfigname}{}
\newenvironment{dfigright}[3][0.5]{%
    \renewcommand{\figname}{#2}
    \renewcommand{\dfigname}{#3}
    \SUBTRACT{1.}{#1}{\wdthb}
    \begin{columns}
        \begin{column}{#1\textwidth}
        }{%
        \end{column}
        \begin{column}{\wdthb\textwidth}
            \includegraphics[width=\textwidth]{\figname}
            \includegraphics[width=\textwidth]{\dfigname}
        \end{column}
    \end{columns}
}



\newenvironment{figleft}[2][0.5]{%
    \SUBTRACT{1.}{#1}{\wdthb}
    \begin{columns}
        \begin{column}{#1\textwidth}
            \includegraphics[width=\textwidth]{#2}
        \end{column}
        \begin{column}{\wdthb\textwidth}
        }{%
        \end{column}
    \end{columns}
}

\newenvironment{dfigleft}[3][0.5]{%
    \SUBTRACT{1.}{#1}{\wdthb}
    \begin{columns}
        \begin{column}{#1\textwidth}
            \includegraphics[width=\textwidth]{#2}
            \includegraphics[width=\textwidth]{#3}
        \end{column}
        \begin{column}{\wdthb\textwidth}
        }{%
        \end{column}
    \end{columns}
}


\newcounter{numimages}

\newenvironment{multifig}[1]{%
    \begin{frame}
        \pdfximage{#1}%
        \setcounter{numimages}{\the\pdflastximagepages}
        \addtocounter{numimages}{-1}

        \begin{tikzpicture}[remember picture, overlay]
            \foreach \pagenum in {1,...,\thenumimages} {%
                \node<handout:0|beamer:\pagenum>[anchor=center] at (current page.center) {
                \includegraphics[width=\textwidth,page=\pagenum]{#1}}; 
            }
            \addtocounter{numimages}{1}
            \node<handout:1|beamer:\thenumimages>[anchor=center] at (current page.center) {
            \includegraphics[width=\textwidth,page=\thenumimages]{#1}}; 
        \end{tikzpicture}
    }{%
    \end{frame}
}


\newcommand{\lik}{\mathcal{L}}
\newcommand{\posterior}{\mathcal{P}}
\newcommand{\prior}{\pi}
\newcommand{\ev}{\mathcal{Z}}

\newcommand{\prob}{\mathrm{P}}


\title{BAMBI\@: Resurrection}
\subtitle{Blind Accelerated Multimodal Bayesian Inference}
\author[Handley] % (optional, for multiple authors)
{Will Handley\\ \small{wh260@cam.ac.uk}}
\institute[University of Cambridge] % (optional)
{%
Astrophysics Group \\
Cavendish Laboratory \\
University of Cambridge
}
\date{Nov 13, 2018}


\begin{document}

\begin{frame}
    \titlepage{}
\end{frame}

\begin{frame}
    \frametitle{Outline}
    \tableofcontents
\end{frame}

\section{Background}

\begin{frame}
    \frametitle{Background}
    \begin{itemize}
        \item Key papers in historical order:
            \begin{description}
                \item[MultiNest] \href{https://arxiv.org/abs/0809.3437}{arXiv:0809.3437} 
                \item[BAMBI]     \href{https://arxiv.org/abs/1110.2997}{arXiv:1110.2997} 
                \item[SkyNet]    \href{https://arxiv.org/abs/1309.0790}{arXiv:1309.0790}
                \item[PolyChord] \href{https://arxiv.org/abs/1506.00171}{arXiv:1506.00171} 
            \end{description}
        \item First widely-successful nested sampling algorithm MultiNest.
        \item BAMBI aims to speed up MultiNest using neural networks.
        \item Neural network approach is developed into stand-alone code SkyNet.
        \item MultiNest is updated from version 1 to version 3.
        \item Higher-dimensional nested sampling available in PolyChord.
    \end{itemize}
\end{frame}


\section{Nested Sampling summary}

\begin{frame}
  \frametitle{Nested Sampling} 
  \begin{itemize}
      \item Completely new approach to sampling:
  \end{itemize}

  Maintain a set $S$ of $n$ samples, which are sequentially updated:

  \begin{description}
    \item[$S_0$:] Generate $n$ samples uniformly over the space.
    \item[$S_{n+1}$:] Delete the lowest probability sample in $S_{n}$, and replace it with a new sample with higher probability
  \end{description}

  \begin{itemize}
      \item This generates a \emph{run} of discarded points.
      \item $n\sim\mathcal{O}(10s-1000s)$
      \item Requires one to be able to uniformly within a region, subject to a {\em hard probability constraint}.
      \item John Skilling's original paper: \href{https://projecteuclid.org/euclid.ba/1340370944}{euclid.ba/1340370944}
  \end{itemize}
\end{frame}

\begin{multifig}{figures/nested_sampling.pdf}
\end{multifig}

\begin{frame}
  \frametitle{How is Nested Sampling used?} 
  \begin{itemize}
      \item Nested sampling generates a \emph{run} of discarded points
      \item These points can be weighted in post-processing to give:
          \begin{itemize}
              \item Posterior samples
              \item Bayesian Evidence (marginal likelihoods)
              \item Kullback Liebler divergence
              \item Partition function
          \end{itemize}
      \item This is possible because the nested sampling scheme is a probabilistic integrator, allowing one to estimate the \emph{density of states}.
  \end{itemize}
\end{frame}

\section{BAMBI}
\begin{frame}
  \frametitle{Neural networks to speed up likelihood calls} 
  \begin{dfigright}[0.65]{./figures/NS1.pdf}{./figures/Artificial_neural_network.png}
    \begin{itemize}
        \item Likelihood calls can often be slow (seconds to minutes in cosmology and particle physics).
        \item Use trained neural network as fast proxy for the likelihood
    \end{itemize}
\end{dfigright}

\end{frame}

\begin{frame}
  \frametitle{BAMBI} 
  \begin{dfigright}[0.65]{./figures/NS1.pdf}{./figures/Artificial_neural_network.png}
    \begin{enumerate}
        \item Every \texttt{updInt/2} iterations, use the last \texttt{updInt} discarded points to train a neural network\label{it:loop1}
        \item Split points $80:20$ training:validation
        \item If accurate network is obtained, replace log-likelihood with NN\@.
        \item Periodically check whether NN remains accurate, and re-train if necesssary
    \end{enumerate}
    \begin{itemize}
        \item Do not use NN in place of likelihood if outside range of training data.
        \item Re-train if most new samples outside 95\% central percentile of training data likelihoods.
    \end{itemize}
\end{dfigright}

\end{frame}

\begin{frame}
  \frametitle{BAMBI\@: good ideas} 
  \begin{dfigright}[0.65]{./figures/NS1.pdf}{./figures/Artificial_neural_network.png}
    \begin{itemize}
        \item At end of procedure, one is left with a set of overlapping NNs.
        \item Each NN capable of predicting the log-likelihood across a certain range.
        \item NNs can be used for further analyses
        \item set of NNs are particularly well-suited to describing posterior peak
    \end{itemize}
\end{dfigright}
\end{frame}

\section{BAMBI\@: Resurrection}
\begin{frame}
  \frametitle{Problems with current BAMBI code} 
    \begin{itemize}
        \item Out-of-date: 
        \item Hard-coded into old MultiNest (v1). 
        \item NN training technology has advanced since 2013.
        \item MultiNest will not be suitable for high-dimensional problems, even with fast NN-proxy likelihoods.
    \end{itemize}

\end{frame}


\begin{frame}
  \frametitle{PyBAMBI} 
    \begin{itemize}
        \item Use latest advances in neural networks.
        \item Use latest versions of PyMultiNest and PyPolyChord.
        \item Use dumper functions to keep everything in python.
        \item Use \href{https://keras.io/}{Keras, TensorFlow and Theano} for NN framework.
    \end{itemize}
\end{frame}

\end{document}
